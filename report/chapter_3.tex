\chapter{Gazebo Simulation Environment for REACSA Relative Navigation}

In this chapter, the setup of the REACSA's sensors within the Gazebo simulation environment is presented. 
\\
The robot is divided into three main modules (see the previous section), and two primary sensors are simulated: a LiDAR and a camera. 
The LiDAR is positioned at a lower level to detect low-lying elements in the environment, while the camera is mounted near the top to provide an overall visual perspective.
\\
For the simulation, the sensor parameters are currently chosen as placeholders for testing purposes. Once the actual hardware is selected, these values should be updated accordingly.

\section{LiDAR Sensor}

The LiDAR is defined as a macro link in the \texttt{xacro} format:

\begin{lstlisting}[basicstyle=\ttfamily\scriptsize]
<xacro:macro name="lidar_link" params="name">
  <link name="${name}">
    <inertial>
      <mass value="0.1"/>
      <origin xyz="0 0 0" rpy="0 0 0"/>
      <inertia ixx="0.000166667" iyy="0.000166667" izz="0.000166667"/>
    </inertial>
    <collision>
      <geometry><box size="0.1 0.1 0.1"/></geometry>
    </collision>
    <visual>
      <geometry><box size="0.1 0.1 0.1"/></geometry>
    </visual>
  </link>
</xacro:macro>
\end{lstlisting}

\noindent The LiDAR sensor is then inserted in Gazebo with basic parameters:

\begin{itemize}
    \item Sensor type: GPU LiDAR
    \item Update rate: 10 Hz
    \item Horizontal scan: 1800 samples over 360°
    \item Vertical scan: 16 samples over ±15°
    \item Range: 0.9 to 100 m (currently reduced for simplicity)
    \item Resolution: 3 cm
\end{itemize}

\begin{lstlisting}[basicstyle=\ttfamily\scriptsize]
<gazebo reference="reacsa_lidar_link">
  <sensor name="lidar_sensor" type="gpu_lidar">
    <topic>lidar</topic>
    <update_rate>10</update_rate>
    <frame>reacsa_lidar_link</frame>
    <lidar>
      <scan>
        <horizontal><samples>1800</samples><min_angle>-3.141593</min_angle><max_angle>3.141593</max_angle></horizontal>
        <vertical><samples>16</samples><min_angle>-0.261799</min_angle><max_angle>0.261799</max_angle></vertical>
      </scan>
      <range><min>0.9</min><max>100</max><resolution>0.03</resolution></range>
    </lidar>
    <alwaysOn>1</alwaysOn>
    <visualize>true</visualize>
  </sensor>
</gazebo>
\end{lstlisting}

\section{Camera Sensor}

The camera is similarly defined as a macro link:

\begin{lstlisting}[basicstyle=\ttfamily\scriptsize]
<xacro:macro name="camera_link" params="name">
  <link name="${name}">
    <inertial><mass value="0.1"/><origin xyz="0 0 0" rpy="0 0 0"/>
      <inertia ixx="0.000166667" iyy="0.000166667" izz="0.000166667"/>
    </inertial>
    <visual><geometry><box size="0.1 0.1 0.1"/></geometry><material name="red"/></visual>
    <collision><geometry><box size="0.1 0.1 0.1"/></geometry></collision>
  </link>
</xacro:macro>
\end{lstlisting}

\noindent In Gazebo, the camera sensor is configured with the following properties:

\begin{itemize}
    \item Sensor type: Camera
    \item Horizontal field of view: 1.047 rad
    \item Image resolution: 320x240 pixels
    \item Clipping range: 0.1–100 m
    \item Update rate: 30 Hz
    \item Visualization enabled
\end{itemize}

\begin{lstlisting}[basicstyle=\ttfamily\scriptsize]
<gazebo reference="camera_link">
  <sensor name="camera_link" type="camera">
    <camera>
      <horizontal_fov>1.047</horizontal_fov>
      <image><width>320</width><height>240</height></image>
      <clip><near>0.1</near><far>100</far></clip>
    </camera>
    <always_on>1</always_on>
    <update_rate>30</update_rate>
    <visualize>true</visualize>
    <topic>camera</topic>
  </sensor>
</gazebo>
\end{lstlisting}

\noindent This setup allows for initial testing of REACSA's perception capabilities in Gazebo, with modular sensor definitions that can be easily replaced once the final hardware is selected.
